%
% This file presents the `chem-acs' style
%
\documentclass[a4paper]{article}
\usepackage[T1]{fontenc}
\usepackage[english,UKenglish,american]{babel}
\usepackage[babel]{csquotes}
\usepackage[
  style=chem-acs,maxnames=111,
  % articletitle,     % To include article titles
  % biblabel=dot,     % Alter bibliography labels
  % chaptertitle,     % Include chapter titles for parts of books
  % pageranges=false, % Only include first page of a range
  % subentry,         % For (a), (b), etc. in sets
  hyperref
  ]{biblatex}
\usepackage[
  colorlinks,
  linkcolor=black,
  urlcolor=black,
  citecolor=black
  ]{hyperref}
\bibliography{biblatex-chem}

\begin{document}

\section*{The \texttt{chem-acs} style}

This style prints numeric citations with bibliography
formatting following the rules of the American Chemical Society,
as outlined in \emph{The ACS Style Guide} \cite{Coghill2006}.
This applies to journals such as \emph{Journal of the American
Chemical Society}, \emph{The Journal of Organic Chemistry}
and \emph{Organometallics}.  Some journals published by the ACS
use superscript citations and some do not: read the package
documentation for more details. With settings for citations as 
given, the citations will be superscript and punctuation will be
moved before citations, for example \autocite{Kabbe1973} or
\autocite{Arduengo1991}.

\nocite{*}

\printbibliography

\end{document}